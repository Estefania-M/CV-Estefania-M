\documentclass[%singlesided,
               doublesided,
               paper=a4,
               fontsize=10pt
              ]{my-resume}


%%%%%%%%%%%%%%%%%%%%%%%%%%%%%%%%%%%%%%%%%%%%%%%%%%%%%%%%%%%%%%%%%%%%%%%%%%%%%%%%
% set geometry
%%%%%%%%%%%%%%%%%%%%%%%%%%%%%%%%%%%%%%%%%%%%%%%%%%%%%%%%%%%%%%%%%%%%%%%%%%%%%%%%

\setlength\highlightwidth{8cm}
\setlength\headerheight{2.5cm}            % note that margintop gets added to this value, i.e. the header bar is 5cm
\setlength\marginleft{1cm}
\setlength\marginright{\marginleft}      % needs to be 1.5 times to be actually equal. why?
\setlength\margintop{0.5cm}
\setlength\marginbottom{1cm}


%%%%%%%%%%%%%%%%%%%%%%%%%%%%%%%%%%%%%%%%%%%%%%%%%%%%%%%%%%%%%%%%%%%%%%%%%%%%%%%%
% FONTS
%%%%%%%%%%%%%%%%%%%%%%%%%%%%%%%%%%%%%%%%%%%%%%%%%%%%%%%%%%%%%%%%%%%%%%%%%%%%%%%%

\RequirePackage{fontspec}
\setmainfont{Carlito}


%%%%%%%%%%%%%%%%%%%%%%%%%%%%%%%%%%%%%%%%%%%%%%%%%%%%%%%%%%%%%%%%%%%%%%%%%%%%%%%%
% COLORS
%%%%%%%%%%%%%%%%%%%%%%%%%%%%%%%%%%%%%%%%%%%%%%%%%%%%%%%%%%%%%%%%%%%%%%%%%%%%%%%%

\colorlet{highlightbarcolor}{lightgray}
\colorlet{headerbarcolor}{darkgray}

\colorlet{headerfontcolor}{white}
\colorlet{accent}{awesome-red}
\colorlet{heading}{black}
\colorlet{emphasis}{black}
\colorlet{body}{black}


%%%%%%%%%%%%%%%%%%%%%%%%%%%%%%%%%%%%%%%%%%%%%%%%%%%%%%%%%%%%%%%%%%%%%%%%%%%%%%%%
% set document
%%%%%%%%%%%%%%%%%%%%%%%%%%%%%%%%%%%%%%%%%%%%%%%%%%%%%%%%%%%%%%%%%%%%%%%%%%%%%%%%


\begin{document}

\name{Estefanía Muñoz Miranda}
\tagline{Estudiante de Ingeniería Civil en Computación, Universidad de Chile.}
\photo[round]{foto.jpg}{\dimexpr \headerheight - \marginbottom}   % make photo exactly match the header with margintop/marginright/marginbottom as margin

\makeheader

\highlightbar{

    \section{Contacto}
    \email{estefania.munoz.m@ug.uchile.cl}
    \phone{+569 9688 8467}
    \github{@Estefania-M}{https://github.com/Estefania-M}
    \linkedin{Estefanía Muñoz Miranda}{https://www.linkedin.com/in/estefania-munoz-m/}
    
    \section{Habilidades Técnicas}
    \skillsection{Lenguajes de Programación}
    \begin{tabular}{r @{\hspace{0.5em}}l}
        {} & \skillsection{{\hspace{0.5em} \footnotesize básico | intermedio| avanzado|}}\\
        \bg{skilllabelcolour}{iconcolour}{python} & \barrule{0.5}{accent}{0.1} \\
        \bg{skilllabelcolour}{iconcolour}{c} & \barrule{0.3}{accent}{0.3} \\
        \bg{skilllabelcolour}{iconcolour}{java} & \barrule{0.15}{accent}{0.45} \\
        \bg{skilllabelcolour}{iconcolour}{html, css} &  \barrule{0.4}{accent}{0.2}\\
        \bg{skilllabelcolour}{iconcolour}{javascript} & \barrule{0.25}{accent}{0.35} \\
        \bg{skilllabelcolour}{iconcolour}{\LaTeX} & \barrule{0.2}{accent}{0.4} \\
        \bg{skilllabelcolour}{iconcolour}{SQL} & \barrule{0.15}{accent}{0.45} \\
    \end{tabular}

    \skillsection{Herramientas de Software}
    \simpleskill{Excel}{Intermedio}
    (\footnotesize{fórmulas y funciones, filtros de búsqueda, gráficos de datos, importación datos csv, etc.}) \vspace{0.5em}\\
    \simpleskill{\normalsize Jupyter Notebook}{Básico}
    (\footnotesize{importación de paquetes, gráficos de datos, lectura de archivos, etc.})\vspace{0.5em}\\
    \simpleskill{\normalsize Visual Studio Code}{Intermedio}
    (\footnotesize{creación de archivos Python, HTML, C, entre otros, importación de paquetes, lectura de archivos, debugging, etc.})\vspace{0.5em}\\
    \simpleskill{\normalsize IntelliJ}{Básico}
    (\footnotesize{creación de archivos Java, importación de paquetes, etc.})\vspace{0.5em}\\
    \simpleskill{\normalsize OpenGL}{Básico}
    (\footnotesize{creación de figuras y objetos 2D y 3D, utilización de texturas y luces, etc.})\\

    \skillsection{\normalsize Idiomas}
    \simpleskill{\normalsize Español}{Nativo}
    \simpleskill{\normalsize Inglés}{Básico}
    \footnotesize(rindiendo curso de inglés en la Universidad)
    
    \section{Proyectos/publicaciones}
    Mis proyectos se encuentran disponibles en  \href{https://github.com/Estefania-M}{GitHub:}\\
    
    \proyect{2022}{\textbf{Página web encargada de agregar viajes y solicitar encargos.} Documento en lenguaje HTML, CSS, JavaScript y Python, Facultad de Ciencias Físicas y Matemáticas, Universidad de Chile.}{https://anakena.dcc.uchile.cl/~cc500209/cgi-bin/Tarea2/cgi-bin/inicio.py} \\
    
    \proyect{2022}{\textbf{Modelamiento de una ciudad típica en OpenGL}. Documento en lenguaje Python, Facultad de Ciencias Físicas y Matemáticas, Universidad de Chile.}{https://github.com/Estefania-M/Modelamiento-Ciudad} \\
    
    \proyect{2021}{\textbf{Las mujeres que sentaron las bases de la astrofísica}. Ensayo Monográfico, Programa Académico de Bachillerato, Universidad de Chile.}{https://github.com/Estefania-M/Monografia}
    
    %\simpleskill{AWS certified cloud practitioner}
    %\simpleskill{AWS certified ML Specialist}
    %\simpleskill{Databricks Lakehouse Platform}

}
\mainbar{
    \section{Perfil Personal}
    Estudiante de ingeniería en búsqueda de desarrollo personal y profesional. Con gran motivación de vivir experiencias nuevas, adquiriendo nuevos aprendizajes. Motivación por el trabajo bien hecho y eficaz. %editar
    
    %\section[\faGears]{Work history}

    \section{Formación Académica}
    \job{2020 - actual}
        {Facultad de Ciencias Físicas y Matemáticas, Universidad de Chile}
        {Estudiante Ingeniería Civil en Computación.}
        {}
    
    \job{Marzo 2019 - Enero 2021}
        {Programa Académico de Bachillerato, Universidad de Chile}
        {Licenciada del Programa Académico de Bachillerato}
        {Grado de Bachiller con Mención en Ciencias Naturales y Exactas.}
        
    \job{2015 - 2018}
        {Liceo Nacional Bicentenario de Excelencia Académica, San Bernardo}
        {Licenciada en Educación Media}
        {Científico Humanista, mención Matemático.}
    
    \section{Actividades Extracurriculares}
    
    \activities{2019 - actual}
        {Universidad de Chile}
        {Monitora en Unidad de Difusión para la Admisión de Pregrado (UDAP)}
        {Realizar charlas institucionales, representar a la universidad y la unidad académica en ferias vocacionales. Apoyo en matrículas, participar de capacitaciones, etc.}
    \activities{Enero 2019}
        {Facultad de Ciencias Físicas y Matemáticas, Universidad de Chile}
        {Estudiante en Escuela de Verano (EdV)}
        {Curso de Matemática IV de duración 1 mes, en donde se evaluaron contenidos principalmente de integrales y ecuaciones diferenciales ordinarias.}
    \activities{2017 - 2018}
        {Pontificia Universidad Católica}
        {Estudiante en Taller de Razonamiento Matemático (TRM) Nivel 1 y 2}
        {Clases una vez a la semana, con desafíos/ejercicios de problemas matemáticos de diferentes contenidos.}
    \activities{Junio 2017}
        {Universidad San Sebastián}
        {Participante en “Triatlón de Matemáticas”}
        {Campeonato matemático en donde los integrantes de cada grupo debían resolver desafíos matemáticos en un determinado tiempo.}
        
    \section{Cursos relevantes}
    \coursesP{Desarrollo de Aplicaciones Web, Universidad de Chile.}{Primavera 2022}
    \coursesP{Bases de Datos, Universidad de Chile.}{Primavera 2022}
    \coursesO{Programación de Software de Sistemas, Universidad de Chile.}{Otoño 2022}
    \coursesO{Modelamiento y Computación Gráfica para Ingenieros, Universidad de Chile.}{Otoño 2022}
    \coursesO{Algoritmos y Estructuras de Datos, Universidad de Chile.}{Otoño 2021}
    
}
\makebody
\clearpage

\end{document}