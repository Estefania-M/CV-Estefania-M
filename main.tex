% Document class and font size
\documentclass[a4paper, 10pt]{extarticle}

% Packages
\usepackage[utf8]{inputenc} % For input encoding
\usepackage{geometry} % For page margins
\geometry{a4paper, margin=0.75in} % Set paper size and margins
\usepackage{titlesec} % For section title formatting
\usepackage{enumitem} % For itemized list formatting
\usepackage{fontawesome5}
\usepackage{parskip}
\usepackage[usenames]{color}
\usepackage{hyperref} % For hyperlinks

% Formatting
\setlist{noitemsep} % Removes item separation
\titleformat{\section}{\large\bfseries}{\thesection}{1em}{}[\titlerule] % Section title format
\titlespacing*{\section}{0pt}{\baselineskip}{\baselineskip} % Section title spacing

% Begin document
\begin{document}

% Disable page numbers
\pagestyle{empty}

% Header
\begin{center}
\textbf{\LARGE ESTEFANÍA MUÑOZ MIRANDA}\\[5pt] % Name
% Contact info
\faPhone ~ +569 9688 8467 ~
% EMAIL
\faEnvelope  ~
\href{mailto:estefania.munoz.m@ug.uchile.cl}{estefania.munoz.m@ug.uchile.cl}  ~
% LINKEDIN
\faLinkedin  ~
\href{https://www.linkedin.com/in/estefania-munoz-m/}{estefania-munoz-m} ~
% GITHUB
\faGithub  ~
\href{https://github.com/Estefania-M}{@Estefania-M} ~
\end{center}

\section*{RESUMEN}
Estudiante de 4to año de Ingeniería Civil en Computación de la Universidad de Chile, en búsqueda de desarrollo personal y profesional. Experiencia en clases particulares y trabajos universitarios de desarrollo web, manejo de datos y desarrollo de algoritmos. Gran motivación por el trabajo bien hecho y eficaz, responsable, puntual, y de rápido aprendizaje.

% Education Section
\section*{EDUCACIÓN}
\noindent
\textbf{Facultad de Ciencias Físicas y Matemáticas, Universidad de Chile} \hfill 2020 | actual\\ % University name and location
Estudiante de Ingeniería Civil en Computación. 

\noindent
\textbf{Programa Académico de Bachillerato, Universidad de Chile} \hfill Marzo 2019 | Enero 2021\\ % University name and location
Licenciada del Programa Académico de Bachillerato.\\
Grado de Bachiller con Mención en Ciencias Naturales y Exactas.

\noindent
\textbf{Liceo Nacional Bicentenario de Excelencia Académica, San Bernardo} \hfill 2015 | 2018\\ % University name and location
Licenciada en Educación Media Científico - Humanista, mención Matemática.

%\section*{EDUCACIÓN}
%\noindent
%\textbf{University Name}, City, State \hfill Enrolled: Month Year | Expected: Month Year\\ % University name and location
%Degree Name \hfill Overall GPA: X.XX | Major GPA X.XX\\ % Degree and GPA
%Threads: Example and Example \hfill Credit standing: Honors % Additional info

%COURSES SECTION
\section*{CURSOS RELEVANTES}
\noindent
Ingeniería de Software, Universidad de Chile. \hfill Otoño 2023\\[2pt]
Desarrollo de Aplicaciones Web, Universidad de Chile. \hfill Primavera 2022\\[2pt]
Bases de Datos, Universidad de Chile. \hfill Primavera 2022\\[2pt]
Programación de Software de Sistemas, Universidad de Chile. \hfill Otoño 2022\\[2pt]
Modelamiento y Computación Gráfica para Ingenieros, Universidad de Chile. \hfill Otoño 2022\\[2pt]
Algoritmos y Estructuras de Datos, Universidad de Chile. \hfill Otoño 2021

% Skills Section
\section*{HABILIDADES TÉCNICAS}
\begin{itemize}
    \item \textbf{Manejo intermedio de Lenguajes de Programación:} Python, C, JavaScript.
    \item \textbf{Manejo básico de Lenguajes de Programación:} Java, SQL, R.
    \item \textbf{Manejo intermedio de Tecnologías Web:} HTML, CSS.
    \item \textbf{Manejo básico de Frameworks:} Django.
    \item \textbf{Uso intermedio de Herramientas de Software:} Excel, Git, RStudio, Visual Studio Code, LaTeX.
    \item \textbf{Uso básico de Herramientas de Software:}  Jupyter Notebook, OpenGL, IntelliJ.
    \item \textbf{Manejo de Idiomas:}  Inglés (básico, rindiendo cursos en la Universidad), Español (Nativo).
\end{itemize}

% Projects Section
\section*{PROYECTOS/PUBLICACIONES}
\noindent
\textbf{Página web con un sistema de reseñas de productos} \textcolor{blue}{\href{https://github.com/Estefania-M/Pagina-Web-Resenas}{\faLink}} \hfill Otoño 2023
\begin{itemize}
    \item Documento en lenguaje Python. Framework utilizada: Django. Basado en Model View Template (MVT). Uso de Git. El usuario puede añadir, editar, eliminar, ver y calificar reseñas dentro del sistema. Proyecto universitario.
\end{itemize}
\noindent
\textbf{Página web para agregar viajes y solicitar encargos} \textcolor{blue}{\href{https://github.com/Estefania-M/Pagina-Web-Encargos-Viajes}{\faLink}} \hfill Primavera 2022
\begin{itemize}
    \item Documento en lenguaje JavaScript, HTML, CSS y Python. Trabajo realizado para curso de la Universidad de Chile.
\end{itemize}

\noindent
\textbf{Modelamiento gráfico de una ciudad típica en OpenGL} \textcolor{blue}{\href{https://github.com/Estefania-M/Modelamiento-Ciudad}{\faLink}} \hfill Otoño 2022
\begin{itemize}
    \item Documento en lenguaje Python. Se utiliza OpenGL para modelar las diferentes figuras y texturas. Trabajo realizado para curso de la Universidad de Chile.
\end{itemize}

\noindent
\textbf{Las mujeres que sentaron las bases
de la astrofísica} \textcolor{blue}{\href{https://github.com/Estefania-M/Monografia}{\faLink}} \hfill Mayo 2020 - Enero 2021
\begin{itemize}
    \item Ensayo Monográfico para optar al del grado de Bachiller en el Programa Académico de Bachillerato, Universidad de Chile.
\end{itemize}

% Experience Section
\section*{ACTIVIDADES EXTRACURRICULARES}
\noindent
\textbf{Clases Particulares} \hfill Región Metropolitana\\
\textit{Tutora} \hfill Marzo 2023 - actual 
\begin{itemize}
    \item Planificación y realización de clases presenciales de contenidos matemáticos a estudiantes de enseñanza media.
\end{itemize}

\noindent
\textbf{Unidad de Admisión y Registros Académicos} \hfill Universidad de Chile\\
\textit{Monitora} \hfill 2019 - actual 
\begin{itemize}
    \item Realización de charlas institucionales, representar a la universidad y a la unidad académica en ferias vocacionales.
    \item Apoyo en matrículas, participación de capacitaciones, realizar y recibir llamados telefónicos de postulantes.
\end{itemize}

\noindent
\textbf{Escuela de Verano (EDV)} \hfill Facultad de Ciencias Físicas y Matemáticas, Universidad de Chile\\ 
\textit{Estudiante} \hfill Enero 2019 
\begin{itemize}
    \item Curso de Matemática IV con duración de 1 mes. Evaluaciones de contenidos matemáticos, principalmente integrales y ecuaciones diferenciales ordinarias.
\end{itemize}

\textbf{Taller de Razonamiento Matemático (TRM) - Nivel 1 y 2} \hfill Pontificia Universidad Católica\\
\textit{Estudiante} \hfill 2017 – 2018 
\begin{itemize}
    \item Clases presenciales una vez a la semana con realización de desafíos/ejercicios de problemas matemáticos de diferentes contenidos. 
\end{itemize}

% End document
\end{document}
